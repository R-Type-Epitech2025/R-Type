\documentclass[12pt, letterpaper]{article}\title{R-TYPE RFC}

\begin{document}
\section{Introduction}
\vspace{5mm} %5mm vertical space

The goal of the R-Type project is to create an online multiplayer copy
of the classic R-Type game (1987).

The key words:  
\begin{itemize}
    \item MUST
    \item MUST NOT
    \item REQUIRED
    \item SHALL
    \item SHALL NOT
    \item SHOULD
    \item RECOMMENDED
    \item MAY
    \item OPTIONAL
\end{itemize}

in this document are to be interpreted as described in RFC
2119.

\begin{enumerate}
\item
  R-Type architecture
\end{enumerate}

The R-Type architecture is a classic client-server game architecture.
All the game engine is in the server. A client connects to the server by
using the R-Type TCP protocol. When connected, the client has the choice
between creating a lobby or joining an existing lobby. Multiple clients
can connect to the server at the same time. Then, a lobby's creator
client can launch a new game. The server can run several games at the
same time.

R-Type UDP Protocol

\begin{verbatim}
Once the game is launched, the client-server communications are done
by using this R-Type UDP Protocol.

Frames transmission

  The first purpose of the R-Type UDP Protocol is to send all the
  frames to display (in the client) from the server to the
  client.

  The server send the sprites informations to the client, then
  the client manage the srites on it's own.

  A SPRITE_LIST is a succession of SPRITE.

  A SPRITE is a succession of 7 numbers representing a sprite to
  display and its position. Each one of these numbers MUST be coded
  on 4 bytes. Thus, a SPRITE MUST be 28 bytes long, and a
  SPRITE_LIST of (n) SPRITE MUST be (n * 28) bytes long.

  The SPRITE composition MUST be the following:

     1st number = the index of the sprite's spritesheet

     2nd number = X coordinate (in pixels) of the sprite's top left
                  corner in its spritesheet

     3rd number = Y coordinate (in pixels) of the sprite's top left
                  corner in its spritesheet

     4th number = the sprite's width (in pixels)

     5th number = the sprite's height (in pixels)

     6th number = X coordinate (in pixels) of the sprite's top left
                  corner in the game window

     7th number = Y coordinate (in pixels) of the sprite's top left
                  corner in the game window
\end{verbatim}

The order of the SPRITE elements in SPRITE\_LIST MUST be the final order
of displaying (in the game, every SPRITE will be displayed over the
previous ones).

The client MUST NOT respond to the server when a payload has been
received.

\begin{verbatim}
Player events

  The player, on the client-side, can execute several actions. On
  each player action, the client MUST send to the server a specific
  event payload containing informations about the action. This
  payload MUST begins with an element (on 4 bytes) from the EVENT
  enumeration, described below:

    enum EVENT {
        MOVE,
        SHOOT,
        QUIT
    };

    The MOVE event:

        The player wants to move (to the left, right, up or down).
        The client MUST add to the payload an element (on 4 bytes)
        of the DIRECTION enumeration (described below) right after
        the EVENT element. Thus, if the player wants to move, the
        payload MUST be 8 bytes long.

        enum DIRECTION {
           LEFT,
           RIGHT,
           UP,
           DOWN
        };

    The SHOOT event:

        The player wants to shoot. In this case, the payload MUST be
        only 4 bytes long.




    The QUIT event:

        The player wants to quit the game. The client MUST warn the
        server with this 4 bytes long payload before quitting. The
        client MAY exit without any response from the server.
\end{verbatim}
\end{document}
